% TEMPLATE for Usenix papers, specifically to meet requirements of
%  USENIX '05
% originally a template for producing IEEE-format articles using LaTeX.
%   written by Matthew Ward, CS Department, Worcester Polytechnic Institute.
% adapted by David Beazley for his excellent SWIG paper in Proceedings,
%   Tcl 96
% turned into a smartass generic template by De Clarke, with thanks to
%   both the above pioneers
% use at your own risk.  Complaints to /dev/null.
% make it two column with no page numbering, default is 10 point
% Munged by Fred Douglis <douglis@research.att.com> 10/97 to separate
% the .sty file from the LaTeX source template, so that people can
% more easily include the .sty file into an existing document.  Also
% changed to more closely follow the style guidelines as represented
% by the Word sample file. 
% Note that since 2010, USENIX does not require endnotes. If you want
% foot of page notes, don't include the endnotes package in the 
% usepackage command, below.
\documentclass[letterpaper,twocolumn,9pt]{article}
\usepackage{usenix,epsfig,endnotes,xspace,listings}
\usepackage{booktabs}
\usepackage{adjustbox}
\usepackage{multirow}
\usepackage{caption}
\usepackage{tabularx}
\usepackage{placeins}
\usepackage{minted}
\lstset{
    numbers=left, % 显示行号
    numberstyle=\tiny, % 行号字体
    keywordstyle=\color{blue!70}, % 关键字颜色
    commentstyle=\color{red!50!green!50!blue!50}, % 注释颜色
    frame=shadowbox, % 为代码块添加阴影框
    rulesepcolor=\color{red!20!green!20!blue!20}, % 阴影框颜色
    xleftmargin=2em, xrightmargin=2em, aboveskip=1em, % 设置代码块的边距
    framexleftmargin=2em % 阴影框左边距
} 
\usepackage[T1]{fontenc}
\usepackage[scaled]{berasans}
% 设置等宽字体为 Consolas 风格
\renewcommand*\ttdefault{txtt}
\newcommand{{\cfos}}{CFOS\xspace}
\begin{document}
%don't want date printed
\date{}
%make title bold and 14 pt font (Latex default is non-bold, 16 pt)
\title{\Large \bf INDENG 174 Group 9 Project Progress Report}
\author{
{\rm Runyuan He} \\
{\rm 3041920716}
\and
{\rm Jiedong Zhang} \\
{\rm 3041913865}
\and
{\rm Qingyang Xu} \\
{\rm 3041979645}
}
\maketitle
% Use the following at camera-ready time to suppress page numbers.
% Comment it out when you first submit the paper for review.
\thispagestyle{empty}

\begin{abstract}
Large Language Model (LLM) inference systems face unique challenges in balancing throughput, latency, and fairness due to highly variable request sizes and complex two-phase execution patterns. This project develops a discrete-event simulation framework to rigorously analyze the interplay between batching strategies and scheduling policies in production LLM serving systems like vLLM and SGLang. We model realistic workloads with non-homogeneous Poisson arrivals and heavy-tailed request distributions, implementing multiple scheduling policies (FCFS, SJF, Predicted-SJF, Priority) and batch processing strategies. Our preliminary results demonstrate that batch size selection critically impacts system performance: moderate batches (32-64) achieve 85\% GPU utilization while maintaining acceptable tail latency, whereas suboptimal configurations degrade p99 latency by 40-60\%. We observe a fundamental tradeoff between scheduling efficiency and fairness, with SJF reducing average latency by 15\% but increasing unfairness (Jain's index: 0.64 vs 0.82 for FCFS). Our planned comprehensive experiments will quantify these tradeoffs across diverse workloads and evaluate adaptive batching strategies for time-varying loads. This work provides the first rigorous, simulation-based analysis of LLM inference scheduling and offers practical configuration guidelines for production deployments.
\end{abstract}

\section{Introduction}

\subsection{Motivation}

The rapid adoption of Large Language Models (LLMs) in production environments has created unprecedented challenges in serving inference requests efficiently. Unlike traditional web services, LLM inference exhibits unique characteristics: highly variable request sizes (10 to 2000+ tokens), unpredictable output lengths, and complex two-phase execution patterns (prefill and decode). These characteristics make traditional queueing theory assumptions inadequate and necessitate specialized analysis.

Real-world LLM serving systems like ChatGPT and Claude experience massive traffic spikes---often 10x increases during peak hours---while users expect sub-second latency for interactive applications. Production deployments using systems like vLLM \cite{kwon2023efficient} and SGLang \cite{zheng2024sglangefficientexecutionstructured} have demonstrated that batching strategies and scheduling policies dramatically impact both throughput and tail latency, yet the optimal configurations remain poorly understood.

The economic stakes are substantial. GPU infrastructure represents the dominant operational cost for LLM services, and even a 10\% improvement in utilization translates to millions of dollars in savings for large-scale deployments. Furthermore, tail latency violations (p99 > 10s) directly degrade user experience and reduce engagement. Our simulation framework addresses this gap by providing rigorous, quantitative analysis of the fundamental tradeoffs between batching efficiency, scheduling fairness, and latency guarantees.

\subsection{Problem Definition and Research Questions}

Building upon our initial proposal, we have refined our focus to address three specific performance bottlenecks in LLM inference systems. Our simulation targets the complex interplay between batching strategies and scheduling policies, with the following research questions:

\textbf{RQ1: Dynamic Batch Size Adaptation.} How should batch sizes be dynamically adjusted based on real-time queue lengths and arrival patterns? The key challenge is determining when the throughput gains from larger batches (improved GPU utilization) outweigh the increased queueing delays (head-of-line blocking). We hypothesize that optimal batch sizes are non-monotonic: too small wastes GPU capacity, while too large introduces excessive delays for short requests.

\textbf{RQ2: Heterogeneous Request Handling.} With requests varying from 10 to 2000+ tokens in prompt length, how do different scheduling policies (FCFS, SJF, priority-based) handle this heterogeneity? Our focus is on minimizing p99 latency while maintaining high throughput and fairness. We investigate whether prioritizing short requests (SJF) causes unacceptable starvation for long requests, and whether aging mechanisms can restore fairness.

\textbf{RQ3: Non-stationary Load Patterns.} Real-world systems experience time-varying loads---diurnal patterns with 10x spikes during peak hours. How do different batching strategies perform under these non-homogeneous Poisson arrivals with rates $\lambda(t)$ varying from 2 to 20 requests/second? We model whether adaptive batching (adjusting batch size based on queue length) outperforms static configurations, particularly during rapid load transitions.

\section{Methodology and Simulation Design}

\subsection{Analytical Approach}

Our approach combines discrete-event simulation with rigorous statistical analysis to model LLM inference servers under realistic workloads. We employ queueing theory fundamentals while extending them to capture LLM-specific characteristics that violate traditional M/M/1 assumptions: (1) highly variable service times due to heterogeneous request sizes, (2) two-phase processing with distinct compute/memory-bound characteristics, and (3) batch-oriented service disciplines.

We adopt a first-principles modeling strategy based on the well-established prefill-decode dichotomy of autoregressive transformers \cite{kwon2023efficient}, deriving simplified yet empirically-grounded timing models that capture essential system behavior. Our simulation framework is implemented in Python using SimPy, a discrete-event simulation library, enabling reproducible experiments with full control over arrival processes, scheduling policies, and batch processing logic.

\subsection{Simulation Framework}
We have successfully implemented a basic discrete-event simulation using SimPy that models an LLM inference server with continuous batching. Our current implementation includes:

\begin{itemize}
\item \textbf{Request Generator:} Implements non-homogeneous Poisson arrivals with configurable rate functions $\lambda(t)$. Currently supports constant, step, and sinusoidal patterns.

\item \textbf{Queue Management:} A priority queue system that can switch between FCFS and SJF policies. Requests are tagged with arrival time, prompt length, and expected output length.

\item \textbf{Batch Processor:} Simulates GPU inference with realistic timing models:
  \begin{itemize}
  \item Prefill time: $T_{prefill} = \alpha \cdot \sum_{i \in batch} l_{prompt,i} + \beta$
  \item Decode time: $T_{decode} = \gamma \cdot \max_{i \in batch} l_{output,i} \cdot |batch|$
  \item Where $\alpha = 0.001$s/token, $\beta = 0.05$s overhead, $\gamma = 0.0005$s/token
  \end{itemize}
\end{itemize}

\subsubsection{Formulation of the LLM Inference Server Model}
For a formal model, we define our server based on the well-established two-phase execution of autoregressive LLM inference~ \cite{kwon2023efficient}.

\begin{enumerate}
    \item \textbf{Phase 1: Prefill (Prompt Processing):} This phase processes the input prompt ($l_{prompt}$) in parallel to generate the KV cache for the first token. This operation is \textbf{compute-bound}, characterized by highly parallelized matrix-matrix operations that effectively saturate GPU compute. As such, its execution time scales approximately linearly with the \textit{total} number of tokens being processed in the batch.

    \item \textbf{Phase 2: Decode (Token Generation):} This phase generates subsequent tokens autoregressively, one at a time. Each step is a \textbf{memory-bound} operation (a matrix-vector operation) that is latency-dominated by data transfer (weights, KV cache) and typically underutilizes GPU compute.
\end{enumerate}

Our simulation captures this dichotomy using a simplified, linear performance model:

\begin{itemize}
    \item \textbf{Prefill Time ($T_{prefill}$):} We model the batch prefill time as:
    $$T_{prefill}=\alpha\cdot\sum_{i\in batch}l_{prompt,i}+\beta$$
    Here, $\alpha$ represents the per-token processing time, reflecting the compute-bound nature of this phase. $\beta$ is a fixed overhead for batch processing setup.

    \item \textbf{Decode Time ($T_{decode\_total}$):} We model the time for the \textit{entire batch} to complete decoding based on iterative steps:
    $$T_{decode\_total} = T_{decode\_step} \times \max_{i \in batch} l_{output, i}$$
    This formula is a key simplification. We assume a single decode \textit{step} (generating one token for all requests in the batch) takes a constant time $T_{decode\_step} = \gamma$. The batch must perform $N = \max_{i \in batch} l_{output, i}$ such steps until the longest request completes.
\end{itemize}

\textbf{Justification of Assumptions:}
Our model, $T_{prefill} \propto \sum l_{prompt}$ and $T_{decode\_total} \propto \max(l_{output})$, is a first-order approximation. We explicitly assume the time per decode step ($\gamma$) is constant. While in reality decode throughput \textit{does} scale with batch size, its performance characteristics remain fundamentally memory-bound and distinct from prefill. This model allows us to capture the essential queueing behavior---where short prefill requests can be blocked by long-running decode batches---which is the central tradeoff studied by systems like vLLM~ \cite{kwon2023efficient} and Sarathi-Serve.

Based on this formulation, our parameters are defined as:
\begin{itemize}
    \item Prefill: $\alpha=0.001$s/token, $\beta=0.05$s overhead.
    \item Decode: $\gamma = T_{decode\_step} = 0.0005$s/step.
\end{itemize}

\subsection{Simulation Architecture and Algorithms}
Our simulation follows an event-driven architecture with three main components:

\textbf{1. Request Generation Module:} Generates requests with prompt lengths from LogNormal($\mu=4, \sigma=1.5$) and output lengths from TruncatedNormal($\mu=100, \sigma=30$). Our choice of a LogNormal distribution is empirically grounded; it is widely used to model heavy-tailed, skewed workloads common in systems~ \cite{arlitt1996web} and specifically for modeling token and request distributions in real-world LLM traces like ShareGPT.

\textbf{2. Scheduling Engine:} Implements multiple policies:
\begin{itemize}
\item FCFS: Simple queue ordering
\item SJF: Sorts by prompt length
\item Predicted-SJF: Estimates total processing time
\end{itemize}

\textbf{3. Batch Processing Pipeline:}
\begin{itemize}
\item Accumulates requests until batch\_size reached or timeout
\item Computes prefill for all prompts in parallel
\item Iteratively generates tokens until all requests complete
\item Tracks per-request and per-batch metrics
\end{itemize}

\subsection{Metrics Collection and Statistical Analysis}

Our simulation framework collects comprehensive metrics for performance evaluation:

\textbf{Latency Metrics:} We track average, median, and percentile latencies (p50, p95, p99) for each request, decomposed into queue wait time and processing time. Tail latencies (p99) are particularly critical for user-facing applications.

\textbf{Throughput and Utilization:} System throughput is measured in requests/second and tokens/second. We compute GPU utilization as the fraction of time spent in active batch processing versus idle waiting.

\textbf{Fairness Metrics:} We employ Jain's fairness index \cite{jain1984quantitative}, computed as $({\sum x_i})^2 / (n \cdot \sum x_i^2)$ where $x_i$ represents per-request latency. This provides a bounded metric (0 to 1) quantifying fairness in resource allocation. Additionally, we measure starvation rate as the percentage of requests exceeding 3x median latency.

\textbf{Statistical Rigor:} Each experiment runs 30 independent replications with different random seeds. We report 95\% confidence intervals using Student's t-distribution and employ a warm-up period of 500 requests to eliminate transient effects. This methodology ensures our results are statistically significant and reproducible.

The simulation maintains detailed logs for each request including arrival time, queue wait time, processing start/end, and total tokens generated, enabling comprehensive performance analysis and post-hoc investigation of anomalies.
\section{Preliminary Results and Impact}

\subsection{Initial Findings}

We have conducted preliminary simulations to validate our framework and obtain early insights into system behavior. These initial experiments used 1000 requests with moderate arrival rates ($\lambda = 5-10$ req/s) and provide directional guidance for our comprehensive experiments.

\subsubsection{Batch Size Sensitivity (Preliminary)}

Our initial tests across batch sizes $B \in \{8, 16, 32, 64\}$ reveal several key insights:

\textbf{Optimal Batch Size for Moderate Load:} For arrival rate $\lambda = 5$ req/s, batch size $B = 32$ provides the best latency-throughput balance. Smaller batch sizes ($B < 16$) underutilize the GPU, achieving only 60-70\% throughput of optimal configurations. Larger batch sizes ($B > 64$) introduce head-of-line blocking, increasing p99 latency by 40-60\%.

\textbf{Queue Buildup Characteristics:} We observe that maximum queue length scales non-linearly with batch size. For $\lambda = 10$ req/s, small batches ($B = 8$) accumulate queues of 400+ requests, while larger batches ($B = 32$) stabilize at 300-350 requests. This suggests a complex relationship between batch processing efficiency and queueing dynamics that warrants deeper investigation.

\textbf{Utilization Plateau:} System utilization (measured as fraction of time in active processing) plateaus at approximately 85\% for batch sizes above 64. This plateau indicates that further increasing batch size yields diminishing returns, as the system becomes bottlenecked by the decode phase for long requests.

\subsubsection{Scheduling Policy Comparison (Preliminary)}

Initial comparisons between FCFS and SJF scheduling policies on heterogeneous workloads reveal significant tradeoffs:

\textbf{Average vs Tail Latency Tradeoff:} FCFS scheduling achieves 15\% lower average latency than SJF, but suffers 40\% higher p99 latency. This occurs because FCFS processes all requests in arrival order, avoiding starvation but failing to optimize for short requests. SJF prioritizes short requests, improving their latency dramatically but causing long requests to queue extensively.

\textbf{Fairness Implications:} Jain's fairness index for FCFS is 0.82, compared to 0.64 for SJF. This quantitatively confirms that SJF introduces significant unfairness. Approximately 8-12\% of long requests in SJF experience starvation (latency > 3x median), suggesting that aging mechanisms or hybrid policies may be necessary for production use.

\subsection{Impact and Implications}

These preliminary results have several important implications for LLM serving systems:

\textbf{Configuration Sensitivity:} The 40-60\% variation in tail latency across batch sizes demonstrates that naive configurations can severely degrade user experience. Operators cannot simply maximize batch size for throughput---careful tuning is essential.

\textbf{Scheduling Policy Selection:} The FCFS vs SJF tradeoff suggests that production systems should consider workload characteristics. For uniform workloads, FCFS suffices. For heterogeneous workloads with SLA requirements on tail latency, hybrid policies with aging or priority classes may be necessary.

\textbf{Economic Impact:} Achieving 85\% GPU utilization versus 60-70\% represents a 20-40\% reduction in required infrastructure for the same throughput. At cloud GPU prices (\$2-5/hour for A100), this translates to substantial cost savings for large-scale deployments.

\textbf{Need for Adaptive Strategies:} The sensitivity to arrival rate and batch size suggests that static configurations are suboptimal for real-world systems with time-varying loads. Our planned experiments on adaptive batching (Experiment 5) will quantify potential improvements.

\subsection{Validation and Limitations}

\textbf{Framework Validation:} We have verified our simulation framework produces expected queueing behavior: (1) queue length increases linearly with arrival rate until saturation, (2) latency distributions match expected heavy-tailed characteristics for LogNormal workloads, and (3) utilization approaches theoretical maximums for stable systems.

\textbf{Current Limitations:} Our preliminary results are based on single-parameter sweeps with limited replications (5-10 runs). The comprehensive experiments (Section~\ref{sec:experiments}) will employ 30 replications and multi-dimensional parameter spaces to establish statistically rigorous conclusions. Additionally, our timing model assumes constant decode step time, which we plan to refine based on empirical measurements from vLLM profiling.

\section{Planned Experiments}
\label{sec:experiments}

\subsection{Experiment Design}

\textbf{Experiment 1: Batch Size Sensitivity Analysis}
\begin{itemize}
\item \textit{Parameters:} Batch sizes $B \in \{1, 2, 4, 8, 16, 32, 64, 128\}$
\item \textit{Fixed conditions:} Constant arrival rate $\lambda = 10$ req/s, FCFS scheduling
\item \textit{Metrics:} Average latency, p50/p95/p99 latency, throughput, GPU utilization
\item \textit{Hypothesis:} Optimal batch size will be between 16-32 for this load level
\end{itemize}

\textbf{Experiment 2: Scheduling Policy Comparison}
\begin{itemize}
\item \textit{Policies:} FCFS, SJF (based on prompt length), Predicted-SJF (estimating output length), Priority-based (SLA-aware)
\item \textit{Workload:} Bimodal distribution - 70\% short requests (10-50 tokens), 30\% long requests (500-2000 tokens)
\item \textit{Metrics:} Fairness index (Jain's index), starvation rate for long requests. We will use the canonical formulation of the fairness index~ \cite{jain1984quantitative} as it is a standard, bounded metric (0 to 1) widely adopted in network and systems scheduling research to quantify fairness in resource allocation~ \cite{mittal2016universal}.
\item \textit{Hypothesis:} SJF will minimize average latency but cause starvation for 5-10\% of long requests
\end{itemize}

\textbf{Experiment 3: Load Stress Testing}
\begin{itemize}
\item \textit{Scenario:} Gradual load increase from $\lambda = 1$ to $\lambda = 50$ req/s
\item \textit{Observation:} Queue length growth, latency degradation curve, saturation point
\item \textit{Goal:} Identify early warning indicators (e.g., queue length > 100) for system overload
\end{itemize}

\textbf{Experiment 4: Workload Distribution Sensitivity}
\begin{itemize}
\item \textit{Distributions tested:}
  \begin{itemize}
  \item Uniform(10, 500) tokens
  \item LogNormal($\mu=4$, $\sigma \in \{0.5, 1.0, 1.5, 2.0\}$)
  \item Power law with $\alpha \in \{1.5, 2.0, 2.5\}$. This range of $\alpha$ (shape parameter) is chosen specifically to test system robustness against heavy-tailed distributions, which are characteristic of web server and network traffic workloads~ \cite{willinger1996bibliographical}. Values of $\alpha$ between 1 and 2 (e.g., 1.5) are known to produce extremely high "burstiness" (infinite variance), while $\alpha > 2$ (e.g., 2.0, 2.5) implies finite variance which provides a rigorous stress test for our scheduling and batching policies.
  \end{itemize}
\item \textit{Analysis:} How robust are optimal batch sizes across distributions?
\end{itemize}

\textbf{Experiment 5: Time-Varying Load Patterns}
\begin{itemize}
\item \textit{Pattern:} Sinusoidal $\lambda(t) = 10 + 8\sin(2\pi t/3600)$ (hourly variation)
\item \textit{Strategies:} Static vs adaptive batch sizing (adjusting every 5 minutes based on queue length)
\item \textit{Expected outcome:} Adaptive batching reduces p99 latency by 20-30\% during peak periods
\end{itemize}

\subsection{Statistical Rigor}
Each experiment will:
\begin{itemize}
\item Run 30 independent replications with different random seeds
\item Report 95\% confidence intervals for all metrics
\item Use warm-up period of 500 requests to reach steady state
\item Collect data from 10,000 requests post-warm-up
\end{itemize}

\subsection{Timeline}
\begin{itemize}
\item Week 1-2: Complete Experiments 1-2 (core functionality validation)
\item Week 3: Run Experiments 3-4 (sensitivity analysis)
\item Week 4: Execute Experiment 5 and compile final results
\end{itemize}
\section{Future Work and Timeline}

\subsection{Remaining Work for Final Report}

Between the draft report and final report deadline, we plan to complete the following tasks:

\subsubsection{Comprehensive Experimental Evaluation}

\textbf{Complete All Five Experiments:} We will execute the full experimental plan outlined in Section~\ref{sec:experiments} with 30 replications each for statistical rigor. This includes:
\begin{itemize}
\item Experiment 1: Full batch size sweep ($B \in \{1, 2, 4, 8, 16, 32, 64, 128\}$) with comprehensive metrics
\item Experiment 2: Comparative evaluation of all four scheduling policies (FCFS, SJF, Predicted-SJF, Priority) on bimodal workloads
\item Experiment 3: Load stress testing to identify saturation points and early warning indicators
\item Experiment 4: Workload distribution sensitivity across Uniform, LogNormal, and PowerLaw distributions
\item Experiment 5: Time-varying load patterns with static vs adaptive batch sizing strategies
\end{itemize}

\textbf{Statistical Analysis and Visualization:} For each experiment, we will generate publication-quality plots showing mean values with 95\% confidence intervals. We will conduct ANOVA tests to determine statistical significance of observed differences and perform sensitivity analysis to validate robustness of findings.

\subsubsection{Model Refinement and Validation}

\textbf{Empirical Validation:} We plan to collect empirical timing measurements from a real vLLM deployment to validate our timing model parameters ($\alpha, \beta, \gamma$). We will compare simulation predictions against actual system behavior to assess model fidelity and refine parameters if necessary.

\textbf{Extended Timing Models:} We will explore more sophisticated timing models that capture batch size dependency in decode phase and potential memory bottlenecks at large batch sizes. This may involve piecewise linear models or empirically-fitted curves based on profiling data.

\subsubsection{Advanced Scheduling Strategies}

\textbf{Adaptive Batching Implementation:} We will implement and evaluate dynamic batch sizing algorithms that adjust based on queue length, arrival rate, and system utilization. Candidate algorithms include:
\begin{itemize}
\item Queue-length-based adaptation: $B(t) = B_{min} + k \cdot Q(t)$ where $Q(t)$ is current queue length
\item Rate-based adaptation: Adjusting batch size based on recent arrival rate estimates
\item Latency-feedback control: Using p99 latency measurements to tune batch size in closed-loop
\end{itemize}

\textbf{Hybrid Scheduling Policies:} We will design and test hybrid policies that combine benefits of FCFS (fairness) and SJF (efficiency). Candidates include SJF with aging, priority classes with round-robin, and weighted fair queueing adapted for LLM workloads.

\subsubsection{Analysis and Interpretation}

\textbf{Queueing Theory Analysis:} We will derive analytical approximations for key metrics (average latency, queue length) using queueing theory, treating the system as a G/G/1 queue with batch service. This will provide theoretical bounds and validate simulation results.

\textbf{Practical Recommendations:} Based on experimental results, we will develop concrete configuration guidelines for production LLM serving systems, including decision trees for selecting batch size and scheduling policy based on workload characteristics and SLA requirements.

\subsection{Timeline and Milestones}

We propose the following timeline for completion:

\textbf{Week 1 (Draft to Week 1 post-draft):}
\begin{itemize}
\item Complete Experiments 1-2 with full 30 replications
\item Conduct empirical validation measurements on vLLM
\item Generate initial plots and statistical analysis
\end{itemize}

\textbf{Week 2 (Week 2 post-draft):}
\begin{itemize}
\item Execute Experiments 3-4
\item Implement adaptive batching algorithms
\item Refine timing model parameters based on empirical data
\end{itemize}

\textbf{Week 3 (Final Week):}
\begin{itemize}
\item Complete Experiment 5 and hybrid scheduling policy evaluation
\item Finalize all statistical analysis and visualizations
\item Write discussion, conclusions, and practical recommendations
\item Prepare final presentation video
\end{itemize}

\subsection{Expected Contributions}

Upon completion, we expect to make the following contributions:

\textbf{Quantitative Analysis of Batching Strategies:} Rigorous, statistically-validated measurements of how batch size impacts latency, throughput, and fairness across diverse workloads. This will provide the first comprehensive characterization of this critical design space for LLM serving.

\textbf{Scheduling Policy Recommendations:} Evidence-based guidelines for selecting scheduling policies based on workload characteristics. We will quantify the fairness-efficiency tradeoff and identify conditions under which each policy is optimal.

\textbf{Adaptive Batching Framework:} Design and evaluation of dynamic batch sizing algorithms that outperform static configurations for time-varying loads. If successful, this could directly improve production LLM serving systems.

\textbf{Open-Source Simulation Framework:} Our simulation framework will be released as open-source software, enabling researchers and practitioners to reproduce our results, test new policies, and analyze their own workloads. The framework is modular, well-documented, and designed for extensibility.

\subsection{Potential Challenges and Mitigation}

\textbf{Computational Resources:} Running 30 replications of 5 experiments with 10,000 requests each requires significant compute time. We estimate 50-100 CPU-hours total. Mitigation: We will parallelize experiments across multiple machines and optimize simulation code for performance.

\textbf{Statistical Significance:} Some effects may be small and require careful statistical analysis to detect. Mitigation: We use 30 replications (sufficient for central limit theorem) and will increase to 50 if initial results show high variance.

\textbf{Model Validity:} Our simplified timing model may not capture all real-world effects. Mitigation: We will validate against empirical measurements and clearly document assumptions and limitations in the final report.


\bibliographystyle{acm}
\bibliography{reference}
\clearpage
\end{document}